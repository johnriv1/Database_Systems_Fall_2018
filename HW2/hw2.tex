\documentclass[11pt]{article}
\usepackage{enumerate}
\usepackage{fancyhdr}
\pagestyle{fancy}
\thispagestyle{empty}
\setlength{\parindent}{0cm}

\setlength{\parskip}{0.3cm plus4mm minus3mm}

\oddsidemargin = 0.0in
\textwidth = 6.5 in
\textheight = 9 in
\headsep = 0in

\lhead{John Rivera, riverj5@rpi.edu}

\title{Database Systems, CSCI 4380-01 Fall 2018 \\
Homework \# 2 Answers}
\date{}
\begin{document}
\maketitle

\vspace*{-0.7in}

\noindent{\bf Homework Statement.} 

\hspace*{0.1in} {\tt students(\underline{rin}, fname, lname, email, optin\_date, optout\_date)} \\
\hspace*{0.1in} {\tt gradables(\underline{gid}, gytype, label, given\_date, due\_date, maxgrade, points, nextg\_id)}\\
\hspace*{0.1in} {\tt grades(\underline{rin, gid}, submission\_date, grade)}


{\bf Question 1.} Write the following queries using relational algebra
using any operator that you wish:

\begin{enumerate}[(a)]

\item Return the RIN of all students who missed a homework that was
  due during their opt-in period. Return the {\tt gid} of the
  corresponding missed homeworks. Remember if there is no opt-out
  date, all homeworks after opt-in date are required.

\begin{eqnarray*}
R1 & = & ((gradables \bowtie grades ) \bowtie students)\\
HW & = & \sigma_{gtype='hw'} R1 \\
Req\_HW & = & \sigma_{optin\_date \neq NULL  \mbox { and } optin\_date< due\_date \mbox { and } (optout\_date = NULL \mbox { or }  optout\_date > due\_date)} HW \\
Answer & = & \pi_{rin, gid} (Req\_HW) - \pi_{rin, gid}(grades)\\
\end{eqnarray*}
  
\item Find the RIN, first and last name of all students who had the
  highest grades for an exam (i.e. gtype {\tt 'exam'} or {\tt
    'finalexam'}). Also return the {\tt gid} and {\tt label} of the
  exams they got the highest grades in.

\begin{eqnarray*}
exam\_grades & = & \pi_{rin, gid, grade}(\sigma_{gtype='exam'}(gradables\bowtie grades)) \\
exam\_grades2[rin2, gid2,grade2] & = &exam\_grades \\
Not\_max & = & \pi_{rin, gid, grade}(\sigma_{grade<grade2}\\
&&(exam\_grades \bowtie_{rin=rin2, gid \neq gid2} exam\_grades2)) \\
Max\_exam\_grades & = & exam\_grades - Not\_max \\
Answer &=& \pi_{rin, fname, lname, gid, label}\\
&&((Max\_exam\_grades \bowtie gradables) \bowtie students )
\end{eqnarray*}

\end{enumerate}

\newpage

{\bf Question 2.} For each of the following new relations:

(1) list all the relevant functional dependencies based on the explanations below, \\
(2) find all keys based on your functional dependencies, \\
(3) discuss whether the relation is in BCNF (Boyce-Codd Normal Form) or not, explain why or why not. \\
(4) discuss whether the relation is in 3NF (Boyce-Codd Normal Form) or not, explain why or why not.

\begin{enumerate} [(a)]

  \item The system keeps track of multiple submissions for the same
  homework gradable like submitty in a relation called {\tt submissions}: 

  {\tt submissions(gid, rin, filename, attemptno, submission\_datetime, isactive, totalruntime)} 

  Each student, gradable and specific attempt corresponds to a
  specific filename. Each filename corresponds to a specific student,
  gradable and attempt. For each filename, there is a specific
  submission\_datetime, isactive value and totalruntime value.
\medskip

(1) $\{ (rin,gid,attemptno\rightarrow filename), (filename \rightarrow rin,gid,attemptno), \\
(filename\rightarrow submission\_datetime,isactive, totalruntime) \}$ \\
(2) keys: (filename), (rin,gid, attemptno)\\ 
(3) both (rin, gid, attemptno) and (filename) are superkeys, so relation is in BCNF \\
(4) in BCNF, so in 3NF


\item Homeworks, quizzes and exams have individual questions. We will
  store the details of grades of each part separately using a
  relation called {\tt grade\_details}:

  {\tt grade\_details(rin, gid, partno, topic, maxpoints, pointsearned)}

  For each gradable (gid) and part, there is a maxpoints value. For
  each gradable, part and student, there is pointsearned. Each
  gradable part may have multiple topics.
\medskip

(1)$\{ (gid,partno\rightarrow maxpoints), (gid,partno,rin \rightarrow pointsearned) \}$ \\
(2) key: (gid, partno, rin, topic) \\
(3) no super keys or trivial dependencies in functional dependencies listed in (1), so not in BCNF \\ 
(4) (gid, partno) and (gid,partno, rin) are not superkeys and none of the f.d's are trivial. Neither maxpoints or pointsearned are prime attributes, so it is not in 3NF

\end{enumerate}

\newpage

{\bf Question 3.} Given the following relation, functional
dependencies and decomposition, answer the following questions:

Relation $R(A,B,C,D,E,F)$ with ${\cal F}= \{AB\rightarrow F, BD\rightarrow C, CE\rightarrow F, F\rightarrow D \}$

Decomposition: $R1(A,B,D)$, $R2(A,B,C,E)$, $R3(B,D,E,F)$

\begin{enumerate}[(a)]

\item Is this decomposition lossless? Show yes or no using Chase decomposition.


\begin{tabular}{|l|l|l|l|l|l|}
\hline
A    & B & C    & D    & E    & F    \\ \hline
a    & b & c1$\,\to\,$c & d    & e1 & f1 \\ \hline
a    & b & c    & d2$\,\to\,$d & e    & f2$\,\to\,$f1$\,\to\,$f \\ \hline
a3 & b & c3$\,\to\,$c1$\,\to\,$c & d    & e    & f    \\ \hline
\end{tabular}
\medskip
\\
Steps: \\
AB$\,\to\,$F, f2$\,\to\,$f1\\
BD$\,\to\,$C, c3 $\,\to\,$c1\\
CE$\,\to\,$F\\
F$\,\to\,$D, d2$\,\to\,$d\\
BD$\,\to\,$C, c1$\,\to\,$c\\
CE$\,\to\,$F, f1$\,\to\,$f


\item Is this decomposition dependency preserving? Show your work.

Note: two show that two sets of functional dependencies, $F_1$ and
$F_2$ are equivalent, it is sufficient to show that (1) all functional
dependencies in $F_1$ are implied by $F_2$, and (2) all all functional
dependencies in $F_2$ are implied by $F_1$.

\begin{tabular}{ll}
R1(A,B,D) & \{AB$\,\to\,$D\}=$F_1$ \\
R2(A,B,C,E) & \{AB$\,\to\,$C\}=$F_2$ \\
R3(B,D,E,F) & \{F$\,\to\,$D\}=$F_3$
\end{tabular}

\begin{eqnarray*}
F_1 \cup F_2 \cup F_3 = \{AB \rightarrow D, AB \rightarrow C, F \rightarrow D\}
\end{eqnarray*}

$F_1$ $\cup$ $F_2$ $\cup$ $F_3$ does not imply CE $\,\to\,$ F, so not dependency preserving


  
\end{enumerate}

\newpage

{\bf Question 4.} Given the following relation, use BCNF decomposition
to convert it to relations in BCNF.

$R(A,B,C,D,E)$ ${\cal F}=\{AB\rightarrow C, C\rightarrow E\}$
\\ \\ \\
Step 1:\\
\medskip
use AB$\,\to\,$C \ \ \ \  AB\textsuperscript{+}= \{A, B, C, E\} \\
R1(A, B, C, E) \ \ \ \ \{AB$\,\to\,$C, C$\,\to\,$E\} \ \ \ \ key:AB $\,\to\,$ Not in BCNF due to C$\,\to\,$E \\
R2(A, B, D) \ \ \ \ \ \ \ \ \{\}  \ \ \ \ \ \ \ \ \ \ \ \ \ \ \ \ \ \ \ \ \ \ \ \  key:ABD$\,\to\,$ in BCNF \\

Step 2:\\
\medskip
C$\,\to\,$E \ \ \ \  C\textsuperscript{+}= \{C, E\} \\
R11(C,E) \ \ \ \ \ \ \ \ \ \ \{C$\,\to\,$E\} \ \ \ \ \ key:C $\,\to\,$ in BCNF\\
R12(A, B, C) \ \ \ \ \{AB$\,\to\,$C\} \ \ \ \ key:AB $\,\to\,$ in BCNF\\
\\ \\
So after BCNF decomposition, we have\\
R2(A, B, D) \\ 
R11(C,E) \\
R12(A, B, C)


\newpage

{\bf Question 5.} Given the following relation, use 3NF decomposition
to convert it to relations in 3NF. For each resulting relation, check
if it is also in BCNF.

$R(A,B,C,D,E,F,G)$ ${\cal F}=\{AB\rightarrow C, CD\rightarrow EF, CF\rightarrow AG \}$
\\ \\

R1(A, B, C) \ \ \ \ \ \ \ \ \ \ \ \ \ \{AB $\,\to\,$ C\} \ \ \ \ \ \ \ \ \ in BCNF (AB is key)\\
\\
R2(C, D, E, F)  \ \ \ \ \ \ \ \ \  \{CD $\,\to\,$ EF\} \ \ \ \ \ \ \ \ \ in BCNF (CD is key)\\
\\
R3(C, F, A, G) \ \ \ \ \ \ \ \ \ \{CF $\,\to\,$ AG\} \ \ \ \ \ \ \ \ \ in BCNF (CF is key)\\
\\
There is no relation with all attributes of a key, so add one\\
\\
R4(A, B, D) \ \ \ \ \ \ \ \ \ \{\} in BCNF \\
\\

{\bf SUBMISSION INSTRUCTIONS.} Submit a PDF document for this homework
using Gradescope. No other format and no hand written homeworks
please. No late submissions will be allowed.

The gradescope for homework submissions will become available by
Tuesday September 18 the latest. 


\end{document}



         
\end{document}

